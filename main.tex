% sage_latex_guidelines.tex V1.10, 24 June 2016

\documentclass[Afour,sageh,times]{sagej}


%%%%%
% variable to include comments or not in the compilation ; set to 1 to include
\def \draft {1}
%\def \draft {0}





% writing utilities

% comments	 and responses
%  -> use this comment to ask questions on what other wrote/answer questions with optional arguments (up to 4 answers)
\usepackage{xparse}
\usepackage{ifthen}
\DeclareDocumentCommand{\comment}{o m o o o o}
{\ifthenelse{\draft=1}{
  \IfValueT{#1}{
      \textcolor{red}{\textbf{C (#1) : }#2}
      \IfValueT{#3}{\textcolor{blue}{\textbf{A1 : }#3}}
      \IfValueT{#4}{\textcolor{ForestGreen}{\textbf{A2 : }#4}}
      \IfValueT{#5}{\textcolor{red!50!blue}{\textbf{A3 : }#5}}
      \IfValueT{#6}{\textcolor{Aquamarine}{\textbf{A4 : }#6}}
    }
    \IfNoValueT{#1}{
      \textcolor{red}{\textbf{C : }#2}
      \IfValueT{#3}{\textcolor{blue}{\textbf{A1 : }#3}}
      \IfValueT{#4}{\textcolor{ForestGreen}{\textbf{A2 : }#4}}
      \IfValueT{#5}{\textcolor{red!50!blue}{\textbf{A3 : }#5}}
      \IfValueT{#6}{\textcolor{Aquamarine}{\textbf{A4 : }#6}}
    }
 }{}
}



% todo
\newcommand{\todo}[1]{
\ifthenelse{\draft=1}{\textcolor{red!50!blue}{\textbf{TODO : \textit{#1}}}}{}
}




\usepackage{moreverb,url}

\usepackage[colorlinks,bookmarksopen,bookmarksnumbered,citecolor=red,urlcolor=red]{hyperref}

\newcommand\BibTeX{{\rmfamily B\kern-.05em \textsc{i\kern-.025em b}\kern-.08em
T\kern-.1667em\lower.7ex\hbox{E}\kern-.125emX}}

\def\volumeyear{2017}

\begin{document}

\runninghead{Le Corre and Raimbault}


\title{
% find title suited to content
Socio-economic Inequalities in Accession to Real Estate Market
}

\author{Thibault Le Corre\affilnum{1} and Juste Raimbault\affilnum{1},\affilnum{2}}

\affiliation{\affilnum{1}UMR CNRS 8504 Geographie-cites\\
\affilnum{2}UMR-T 9403 IFSTTAR LVMT}

\corrauth{Thibault Le Corre,
UMR CNRS 8504 Geographie-cites,
13 rue du Four, 
75006 Paris, FR.}

\email{t.lecorre@parisgeo.cnrs.fr}

\begin{abstract}
We study socio-economic drivers of accession to property in Real Estate Market, based on the Greater Paris case. Our contribution is twofold : (\textit{some empirical/theoretical results}), and methodological by introducing an heuristic to reconstruct missing data by using kernel mixture extrapolation.
\end{abstract}

\keywords{Real Estate Market, Data Extrapolation}

\maketitle







%%%%%%%%%%%%%%%%%%%%%%
\section{Introduction}
%%%%%%%%%%%%%%%%%%%%%%

In most capitalist advanced countries, housing prices increases faster than households incomes. In some of them, more exposed with toxic financial assets, 2007's subprimes crisis lead to a momentary burst, but in others the bubble stayed ``robust''~\cite{timbeau2013bulles}.Starting from the bubble explanation and haunted by the notion of equilibrium, hortodox eonomists see in inflation an existing unbalanced stock between housing supply and households demand. In that terms macro-prices dynamics can not be understood by their models and finally put the emphasises on the unresponsability of individuals households, adopting a speculative behaviour similar to financial stakeholders. As if housing was an asset like the others and we would buy one only for the exchange value.
More recently, litterature on political economy of housing give us the tends of the public policy of housing. In that terms it is became common to accept housing "as a pillar of the capitalist economy" (\cite{aalbers2016financialization}, page 10)

-comparative studies : dette et prix
-critique des modèles comparatifs : une vision parfois romantique qui ommet les tendances générales : privatisation du stock + libéralisation des politiques de régulation du financement
--> Le rôle du logement dans l'économie
-Critique de la logique de la formation des prix par des "acteurs de la promotion" + le remplacement social (vision écologiste)
Often, marxist literature investigating on housing rising prices put the emphasises on the role of real estate developers. One of the most famous theory has been developed by Neil Smith through the concept of rent-gap model. We do not assume that this model is unefficient but it seems that it can not explain the contemporain prices formation in entire metropolitan urban regions for one major reason. The housing market is not feed by new constructions but by the circulation of used housing. Moreover, research on real estate developpers have shown their spatial diversification after the inflation. In that sense they are attentive to the "markets signals" more than market markers. However we do accept the analytical concept of devaluation. But in our perspective until nowadays, massive devaluation of real estate stock has been avoid thanks to the  More generally the logic 

PB : Dans quelle mesure la hausse du prix immobilier est le résulat de l'évolution des conditions d'octroie du crédit immobilier? 
les ratio LTV dans les espaces socio correpond t-ils au capital empruntable selon les conditions d'octroie? Quels sont les "régimes" immobiliers des espaces urbains? (un régime patrimonial , un régime par le crédit...)?
Comment peut-on mesurer la structuration sptiale des marchés immobiliers au niveau d'une région urbaine? Que détermine cette structuration en termes de régimes de marchés?

--> Politiques de scoring : quantification du risque

Since Halbwachs seminal study~\cite{halbwachs1913classe}, urban rese
archers have tended to explain socio-economic inequality as  a  function  of  housing  affordability  (i.e.  the  price to
income  ratio).  Few  have  explicitly  attempted  to  link socio spatial  inequality  to  the  financial  dynamics  of  residential  housing

%%%%%%%%%%%%%%%%%%%%%%
\section{Residential real estate studying approach}
%%%%%%%%%%%%%%%%%%%%%%
%%%%%%%%%%%%%%%%%%%%%%
\subsection{Prices as markets segmentation reveals}
%%%%%%%%%%%%%%%%%%%%%%

--> Mesure de l'autocorrélation spatiale des prix : une première idée des structures existantes

\todo[JR]{faire un moran dans le temps et avec different spans pour voir quel niveau de non-stationarité on a, à la fois dans le temps et dans l'espace ; manière supplémentaire de justifier notre approche}

%%%%%%%%%%%%%%%%%%%%%%
\subsection{Why hedonic prices models are dangerous?}
%%%%%%%%%%%%%%%%%%%%%%

%%%%%%%%%%%%%%%%%%%%%%
\subsection{A socio-economic households approach}
%%%%%%%%%%%%%%%%%%%%%%

%%%%%%%%%%%%%%%%%%%%%%
\section{Methodological implication and data}
%%%%%%%%%%%%%%%%%%%%%%


\textbf{Objectives : }
\begin{enumerate}
\item Obtain a spatio-temporal typology of local markets, by linking transactions properties with local structural socio-economic properties
\item Link it with capital and debts characteristic, as an external validation and application of the typology 
\end{enumerate}

%%%%%%%%%%%%%%%%%%%%%%
\section{Results}
%%%%%%%%%%%%%%%%%%%%%%
%%%%%%%%%%%%%%%%%%%%%%
\section{Data}
%%%%%%%%%%%%%%%%%%%%%%

\textit{Description of various datasets, variables, granularity, time span etc.}


We use the following datasets : 
\begin{itemize}
\item BIEN Database : spatialized real estate transactions, includes some socio-economic information on buyer, granularity : IRIS
\item Census : economic structure (median income by CSP) to be extrapolated
\item Patrimoine database : non-spatialized individual informations such as private capital, debts
\item Zero-rate Loans
\end{itemize}


%%%%%%%%%%%%%%%%%%%%%%
\section{General Methodology}
%%%%%%%%%%%%%%%%%%%%%%

To pursue our research objectives, we apply an iterative methodology with the following steps :
\begin{enumerate}
\item Extrapolate (through reverse kernel extrapolation) the socio-economic structure at the iris level
\item Using unsupervised data-mining, establish spatio-temporal typologies using (i) real estate transactions alone ; (ii) socio-economic structure alone ; (iii) coupled real estate transactions and socio-economic structure. At this stage different techniques have to be tested, for example for the role of space by using GWR to extract an optimal spatial scale. Testing the influence of second order relations (correlations) on the typologies is also an option.
\item Using the patrimoine database, establish a relation (with regressions e.g.) between explicative variables present in BIEN, PTZ and Patrimoine database (and that are comparable) and explicated variables present in patrimoine only. To be spatially applicable, the conditioning on explicative variable must be sufficient to a certain level to explain spatial variations (local markets) - this point could be checked on BIEN only using GWR. The regressed functions can then be used to extrapolate explicated variable spatially, and better understand the nature of local markets and in a way externally validate the typology.
\end{enumerate}





%%%%%%%%%%%%%%%%%%%%%%
\section{Data Extrapolation}
%%%%%%%%%%%%%%%%%%%%%%

When using aggregated statistical data, one often observe an aggregated distribution over a classification. We propose a desaggregation method under the assumption of a kernel mixture.

\cite{pivano2016desagreg} : recent phd thesis aiming at desagregating mobility data


When studying geographical data, aggregation is generally done on different aspects. Spatial and temporal aggregations have been widely studied (MAUP, etc.). The aggregation over classes of a population may be problematic : for example, one can have the distribution of incomes for a population of a geographical area, but not the distributions conditioned to socio-economic classes (CSP) that are crucial for some statistical analyses. We describe here a method to reconstruct these, under the knowledge of population composition and simplifying assumptions on the shape of the conditional distributions. This problem is close to inverse problems in various fields of applied statistics, and using Gaussian mixtures is a well-known method~\cite{yu2012solving}. It can also be understood as mode finding with a fixed number of modes~\cite{carreira2000mode}.


%%%%%%%%%%%%%%%%%%%%%%
\subsection{Formalization}

% put the formalization in appendix, describe with words here.


\paragraph{General case}

We assume a random variable $W$ describing a population, for which we know the empirical distribution with density $f(w)$. The population is stratified into a finite number of categories $c\in C$. We assume that the shape of the distribution is known for each category and that it can be expressed with a kernel $k_c (w)$. With categories yielding a partition of the population, we have directly by independence, where $w_c$ is the proportion of category $c$ in the population such that $\sum_{c\in C} w_c = 1$,

\[
f(w) = \sum_{c\in C} w_c \cdot k_c (w)
\]

\paragraph{Parametrization}

Kernels are the unknown to be determined from the empirical distribution. In the general case, we want to minimize a cost function between the two distribution, i.e. solving
\[\min_{k_c} K(f,\sum_{c\in C} w_c k_c)\]

Let simplify the problem and take similar distributions for each category, such that $k_c = g(\vec{\alpha}_c)$. With a simple mean-square error cost function, our optimization problem becomes

\[
\min_{\vec{\alpha}_c} \int_w \left(f(w) - \sum_c w_c \cdot \left[ g(\vec{\alpha}_c)\right] (w) \right)^2 dw
\]


%%%%%%%%%%%%%%%%%%%%%%
\subsection{Implementation}

\textit{Implement the optimization problem in R}

In practice, we will have an histogram $\left(f(x_j)\right)_j$ of the empirical distribution, yielding a discrete mean-square integration. The implemented function will be of the form

\texttt{extrapolate(distrib,weights,kernel,bounds)}

with \texttt{distrib} the histogram, \texttt{weights} composition of the population,\texttt{kernel} the kernel function (taking parameters and variable as arguments), \texttt{bounds} boundary for parameters (should be tunable for each category or general).

\medskip

\textbf{Remarks :}
\begin{itemize}
\item kernel type can be left generic
\item in what case do we have a convex problem (yielding unicity and speed of resolution) ?
\end{itemize}


%%%%%%%%%%%%%%%%%%%%%%
\subsection{Validation}

\subsubsection{Sensitivity Analysis}



\subsubsection{External Validation}

zero rate mortgages : for a subpopulation, data we extrapolate is given (beware of selection bias) $\rightarrow$ validate on it.



%%%%%%%%%%%%%%%%%%%%%%
\section{Typology of Local Markets}
%%%%%%%%%%%%%%%%%%%%%%


\subsection{Context}

Theoretical considerations and literature review suggest that housing market is segmented into at least three distinct categories, that should furthermore correspond to clearly identified spatial entities. The so-called Market Regimes are :

\begin{itemize}
\item Periurban
\item Saturated markets for middle-income classes (usage market)
\item Capitalisation markets
\end{itemize}


\subsection{Methodology}


The aim is to identify spatial regimes. These should answer differently to perturbations as the 2008 crisis, what is a mean of external validation.

The classification can be done at the first, but also second order (correlations). Indeed, if households have a large range of choices that are effectively realized, the correlation between socio-economic and transaction variables should vanish.

The type of dwelling (independently from surface) should be crucial for the question of choice.

The geographical provenance of buyers should also contain some signal, as capitalisation can occur far from the residence, whereas modest household will tend to stay in local range.


Using GWR to extract an optimal spatial scale is also a possibility.

\paragraph{Technique used}

\todo{check the correlation structure / effective dimension of data, to see if standard clustering is enough or if we need to do some Random Forest Classification.}


\subsection{Classification}

\textit{classification in itself}


\subsection{Sensitivity Analysis}

\todo{do a minimal sensitivity analysis of classification to errors in extrapolated data, to see if the error in the previous stage can have a strong influence (it shouldn't, even locally if we use appropriate spatial operators before classifying) ; think of an indicator/a way to give an upper bound on this uncertainty ?}

%%%%%%%%%%%%%%%%%%%%%%
\section{Conclusion}
%%%%%%%%%%%%%%%%%%%%%%



% acknowledgements
%\begin{acks}
%\end{acks}

% end notes of needed
%\theendnotes


%%%%%
%% Biblio

\bibliographystyle{SageH}
\bibliography{biblio.bib} 


%\begin{thebibliography}{99}
%\end{thebibliography}



% supplementary materials

%\begin{sm}
%\end{sm}


\end{document}


%%%%%%%%%%%%%%%%%
%% Templates
%%%%%%%%%%%%%%%%%


%\begin{table}[h]
%\small\sf\centering
%\caption{The choice of options.\label{T1}}
%\begin{tabular}{lll}
%\toprule
%Option&Trim and font size&Columns\\
%\midrule
%\texttt{shortAfour}& 210 $\times$ 280 mm, 10pt& Double column\\
%\texttt{Afour} &210 $\times$ 297 mm, 10pt& Double column\\
%\texttt{MCfour} &189 $\times$ 246 mm, 10pt& Double column\\
%\texttt{PCfour} &170 $\times$ 242 mm, 10pt& Double column\\
%\texttt{Royal} &156 $\times$ 234 mm, 10pt& Single column\\
%\texttt{Crown} &7.25 $\times$ 9.5 in, 10pt&Single column\\
%\texttt{Review} & 156 $\times$ 234 mm, 12pt & Single column\\
%\bottomrule
%\end{tabular}\\[10pt]
%\begin{tabular}{ll}
%\toprule
%Option&Reference style\\
%\midrule
%\texttt{sageh}&SAGE Harvard style (author-year)\\
%\texttt{sagev}&SAGE Vancouver style (superscript numbers)\\
%\texttt{sageapa}&APA style (author-year)\\
%\bottomrule
%\end{tabular}
%\end{table}




